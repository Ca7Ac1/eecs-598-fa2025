\documentclass[12pt]{article}

%AMS-TeX packages
\usepackage{amssymb,amsmath,amsthm}
%geometry (sets margin) and other useful packages
\usepackage[margin=1.25in]{geometry}
\usepackage{tikz}
\usepackage{tikz-cd}
\usepackage{graphicx,ctable,booktabs}
\usepackage{mathpartir}

\usepackage[sort&compress,square,comma,authoryear]{natbib}
\bibliographystyle{plainnat}

%
%Redefining sections as problems
%
\makeatletter
\newtheorem{lemma}{Lemma}
\newenvironment{problem}{\@startsection
       {section}
       {1}
       {-.2em}
       {-3.5ex plus -1ex minus -.2ex}
       {2.3ex plus .2ex}
       {\pagebreak[3]%forces pagebreak when space is small; use \eject for better results
       \large\bf\noindent{Problem }
       }
       }
       {%\vspace{1ex}\begin{center} \rule{0.3\linewidth}{.3pt}\end{center}}
       \begin{center}\large\bf \ldots\ldots\ldots\end{center}}
\makeatother


%
%Fancy-header package to modify header/page numbering
%
\usepackage{fancyhdr}
\pagestyle{fancy}
%\addtolength{\headwidth}{\marginparsep} %these change header-rule width
%\addtolength{\headwidth}{\marginparwidth}
\lhead{Problem \thesection}
\chead{}
\rhead{\thepage}
\lfoot{\small\scshape EECS 598: Category Theory}
\cfoot{}
\rfoot{\footnotesize PS 1}
\renewcommand{\headrulewidth}{.3pt}
\renewcommand{\footrulewidth}{.3pt}
\setlength\voffset{-0.25in}
\setlength\textheight{648pt}

%%%%%%%%%%%%%%%%%%%%%%%%%%%%%%%%%%%%%%%%%%%%%%%

%
%Contents of problem set
%

\newcommand{\meet}{\wedge}
\newcommand{\join}{\vee}
\newcommand{\iplmeets}{\textrm{IPL}(\top,\meet)}
\newcommand{\iplneg}{\textrm{IPL}(\top,\meet,\supset)}
\newcommand{\downset}{\mathcal P_{\downarrow}}
\newcommand{\down}{{\downarrow}}

\begin{document}

\title{Problem Set 1: Logic and Order Theory}
\date{Released: August 28, 2025\\ Due: September 11, 2025, 11:59pm}
\maketitle

Submit your solutions to this homework on Canvas alone or in a group of 2 or
3. Your solutions must be submitted in pdf produced using LaTeX.

If you haven't already, sign up to scribe and present homework
solutions on the course github repo.

%% \begin{definition}
%%   A Heyting Algebra $X$ is a preordered set equipped with a top
%%   element and for any two elements $x, y \in X$, a meet $x \wedge y$
%%   and an implication $x \supset y$.

%%   A biHeyting Algebra is a Heyting algebra that further is equipped
%%   with a least element and for any two elements $x,y$ a join $x \vee
%%   y$.
%% \end{definition}

\begin{problem}{Distributivity}
  A lattice (poset with finite meets and joins) is \emph{distributive}
  when binary meets and joins satisfy a distributive law:
  \[ x \wedge (y \vee z) = (x \wedge y) \vee (x \wedge z) \]

  We say that propositions $A$ and $B$ of a logic are
  \emph{equivalent} when the judgments $A \vdash B$ and $B \vdash A$
  are both provable.

  \begin{enumerate}
  \item Show that IPL satisfies this distributive law in that for any
    propositions $A, B, C$, the propositions $A \wedge (B \vee C)$ and
    $(A \wedge B) \vee (A \wedge C)$ are equivalent.
  \item Show that any biHeyting algebra (poset with finite meets, finite joins
    and an implication operation) is a distributive lattice.
  \end{enumerate}
\end{problem}

\begin{problem}{Intuitionistic and Classical Logic}
  The law of excluded middle or principle of omniscience is the
  following axiom scheme: for all propositions $A$ the axiom
  \[
  \inferrule
  {~}
  {\Gamma \vdash A \vee \neg A}
  \]
  The law of double negation elimination is the axiom scheme
  \[
  \inferrule
  {~}
  {\Gamma \vdash \neg(\neg A) \supset A}
  \]

  Since the first non-boolean model of IPL that we saw is a 3 element
  Heyting algebra, it is easy to get the impression that
  intuitionistic logic is about ``multi-valued logics'' where there
  are some ``intermediate'' truth value other than just true and
  false. But this is misleading. Firstly, there are many boolean
  algebras with more than 2 elements (the powerset of any set) that
  are models of classical logic. Secondly, while there are more than 2
  elements in a model, \emph{within} the logic, we can never separate
  any proposition from true and false.

  \begin{enumerate}
  \item Show that in IPL extended with the law of excluded middle, the
    law of double negation elimination is admissible and vice-versa.
  \item The following might be called the ``intuitionistic law of
    excluded middle'', for all $\Gamma, A$:
    \[ \inferrule*{~}{\Gamma \vdash \neg (\neg (A \iff \top) \wedge \neg (A \iff \bot))} \]
    Intuitively this says ``no proposition is not equivalent to true
    and not equivalent to false'', where we are using the notations
    $\neg B = (B \supset \bot)$ and $B \iff C = (B \supset C) \wedge
    (C \supset B)$.

    Show that the intuitionistic law of excluded middle is derivable
    for all $A$ in IPL. A full proof tree for this will be quite
    large, so I encourage you to develop intermediate reasoning
    principles to make this proof clearer.
  \end{enumerate}
\end{problem}

\begin{problem}{Conservativity Results}
  Fix a signature with propositional variables, but no axioms. In this
  problem our goal is to prove that $\iplneg$ is a \emph{conservative
  extension} of $\iplmeets$\footnote{This conservativity result also
  holds when both have disjunction, but the proof is slightly more
  complex}. That is, we will show that for any judgment $\Gamma \vdash
  A$ where the propositions in $\Gamma, A$ are generated using only
  $\top, \wedge$ and propositional variables, if there is a proof that
  uses the implication $\supset$ then there is a proof that doesn't
  use it. In other words, $\supset$ doesn't let us prove anything new
  about propositions that don't involve $\supset$. This means that we
  can use the richer logic of $\iplneg$ to prove results that hold in
  any poset with finite meets, even those that don't support an
  implication structure. And while $\supset$ can't allow us to prove
  anything new, it might allow us to write a \emph{shorter} or more
  \emph{intuitive} proof.

  To prove conservativity, we will show that the inclusion (a monotone
  function of posets) $i : \iplmeets \to \iplneg$ is an \emph{order
  embedding}: if $i(A) \leq i(B)$ then $A \leq B$. Recall that the
  ordering here is provability of the hypothetical judgment $\vdash$,
  so this means if $A \vdash B$ in $\iplneg$ then $A \vdash B$ in
  $\iplmeets$.

  Key to this proof is the \emph{initiality} property of each variant of IPL:
  \begin{itemize}
  \item For any poset $P$ with finite meets and an assignment
    $\sigma(X) \in P$ for each propositional variable, there is a
    \emph{unique} monotone function $\overline \sigma : \iplmeets \to P$
    that preserves finite meets and respects the assignment of
    propositional variables $\overline \sigma(X) = \sigma(X)$.
  \item An analogous property holds for $\iplneg$
    but the poset $P$ must have an implication as well and $\overline
    \sigma$ is the unique monotone function preserving finite meets and implication that
    respects the assignment $\sigma$.
  \end{itemize}

  \begin{enumerate}
  \item First, show that for any preorder $Q$, the set of downward-closed
    subsets $\downset Q$ with subset inclusion as ordering
    is a poset and a Heyting algebra.
  \item Next, for any preorder $Q$ we can define a function $\down : Q \to \downset Q$ defined by $\down(x) = \{ y \,|\, y \leq x \}$.
    \begin{itemize}
    \item Show that $\down : Q \to \downset Q$ is monotone.
    \item Show that $\down$ is an order embedding: if $\down(x) \leq \down(y)$ then $x \leq y$.
    \item Show that $\down$ preserves finite meets that exist\footnote{though we do not need it for this problem, it also preserves any implications that exist}.
    \end{itemize}
  \item Show that if $i : P \to Q$ and $j : Q \to R$ are monotone
    functions and $j \circ i : P \to R$ is an order embedding then $i$
    is an order embedding.
  \item Use the initiality property of $\iplneg$ to
    construct a monotone function $f : \iplneg \to
    {\downset \iplneg}$ that makes the following diagram
    commute:
    % https://q.uiver.app/?q=WzAsMyxbMCwxLCJJUEwoXFx0b3AsXFx3ZWRnZSkiXSxbMiwxLCJJUEwoXFx0b3AsXFx3ZWRnZSxcXHN1cHNldCkiXSxbMSwwLCJCb29sXntJUEwoXFx0b3AsIFxcd2VkZ2UpXm99Il0sWzAsMiwiWSJdLFswLDEsImkiLDJdLFsxLDIsIlxcb3ZlcmxpbmVcXHNpZ21hIiwyXV0=
    \[\begin{tikzcd}
	& {\downset {\iplmeets}} \\
	{\iplmeets} && {\iplneg}
	\arrow["\down", from=2-1, to=1-2]
	\arrow["i"', from=2-1, to=2-3]
	\arrow["f"', from=2-3, to=1-2]
    \end{tikzcd}\]
    Note that you will need to use the initiality of $\iplmeets$
    to show the diagram commutes.
  \end{enumerate}
  Then since $\down$ is an embedding, we conclude that the inclusion $i$ is an
  embedding.
\end{problem}

\end{document}
