\documentclass[12pt]{article}

%AMS-TeX packages
\usepackage{amssymb,amsmath,amsthm}
%geometry (sets margin) and other useful packages
\usepackage[margin=1.25in]{geometry}
\usepackage{tikz}
\usepackage{tikz-cd}
\usepackage{graphicx,ctable,booktabs}
\usepackage{mathpartir}
\usepackage{minted}

\usepackage[sort&compress,square,comma,authoryear]{natbib}
\bibliographystyle{plainnat}

%
%Redefining sections as problems
%
\makeatletter
\newtheorem{lemma}{Lemma}
\newenvironment{problem}{\@startsection
       {section}
       {1}
       {-.2em}
       {-3.5ex plus -1ex minus -.2ex}
       {2.3ex plus .2ex}
       {\pagebreak[3]%forces pagebreak when space is small; use \eject for better results
       \large\bf\noindent{Problem }
       }
       }
       {%\vspace{1ex}\begin{center} \rule{0.3\linewidth}{.3pt}\end{center}}
       \begin{center}\large\bf \ldots\ldots\ldots\end{center}}
\makeatother


%
%Fancy-header package to modify header/page numbering
%
\usepackage{fancyhdr}
\pagestyle{fancy}
%\addtolength{\headwidth}{\marginparsep} %these change header-rule width
%\addtolength{\headwidth}{\marginparwidth}
%% \lhead{Problem \thesection}
\chead{}
\rhead{\thepage}
\lfoot{\small\scshape EECS 598: Category Theory}
\cfoot{}
\rfoot{\footnotesize PS 1}
\renewcommand{\headrulewidth}{.3pt}
\renewcommand{\footrulewidth}{.3pt}
\setlength\voffset{-0.25in}
\setlength\textheight{648pt}

%%%%%%%%%%%%%%%%%%%%%%%%%%%%%%%%%%%%%%%%%%%%%%%

%
%Contents of problem set
%

\newcommand{\meet}{\wedge}
\newcommand{\join}{\vee}
\newcommand{\iplmeets}{\textrm{IPL}(\top,\meet)}
\newcommand{\iplneg}{\textrm{IPL}(\top,\meet,\supset)}
\newcommand{\downset}{\mathcal P_{\downarrow}}
\newcommand{\down}{{\downarrow}}

\newcommand{\Set}{\textrm{Set}}
\newcommand{\casePlus}[5]{\textrm{case}_{+}\,{#1}\{\sigma_1{#2}\to {#3}|\sigma_2{#4}\to {#5}\}}
\newcommand{\caseZero}[1]{\textrm{case}_0\,{#1}\{\}}
\newcommand{\id}{\textrm{id}}
\newcommand{\lfpt}{\Sigma_{\textrm{lfpt}}}

\newcommand{\cat}{\mathcal}

\begin{document}

\title{Problem Set 3: Functoriality and Naturality}
\date{Released: September 25, 2025\\
  %% Updated ??
  Due: October 9, 2025, 11:59pm
}
\maketitle

Submit your solutions to this homework on Canvas alone or in a group of 2.
Your solutions must be submitted in pdf produced using LaTeX.

\begin{problem}{Bifunctors}
  Functors provide a notion of single-argument morphism between
  categories. A \emph{bifunctor} from $\cat C$ and $\cat D$ to $\cat
  E$ is a notion of ``functor of two arguments''. In this problem we
  will show that various ways of defining a bifunctor are equivalent.

  \begin{itemize}
  \item A \emph{jointly functorial} bifunctor $F : \cat C , \cat D \to \cat E$ consists of
    \begin{enumerate}
    \item An action on objects $F_0 : \cat C_0 \times \cat D_0 \to \cat E_0$
    \item A joint action on morphisms, which for every $A,A' \in \cat C_0$ and $B , B' \in \cat D_0$ gives a function $F_1 : \cat C_1(A,A')\times \cat D_1(B,B') \to \cat E_1(F_0(A,B), F_0(A',B'))$
    \item satisfying joint functoriality laws that $F_1(\id_A,\id_B) = \id_{F_0(A,B)}$ and $F_1(f \circ f', g \circ g') = F_1(f,g) \circ F_1(f',g')$
    \end{enumerate}
  \item A \emph{separately functorial} bifunctor $F : \cat C , \cat D \to \cat E$ consists of
    \begin{enumerate}
    \item An action on objects $F_0 : \cat C_0 \times \cat D_0 \to \cat E_0$
    \item A \emph{left} action on morphisms, which for every $A,A' \in \cat C_0$ and $B \in \cat D_0$ gives a function $F_l : \cat C_1(A,A') \to \cat E_1(F_0(A,B), F_0(A',B))$
    \item A \emph{right} action on morhpisms, which for every $A \in \cat C_0$ and $B,B' \in \cat D_0$ gives a function $F_r : \cat D_1(B,B') \to \cat E_1(F_0(A,B), F_0(A,B'))$
    \item Satisfying left functoriality laws $F_l(\id_A) = \id_{F_0(A,B)}$ and $F_l(f \circ f') = F_l(f) \circ F_l(f')$ 
    \item Satisfying right functoriality laws $F_r(\id_B) = \id_{F_0(A,B)}$ and $F_r(g \circ g') = F_r(g) \circ F_r(g')$
    \item Satisfying a commutative law $F_l(f) \circ F_r(g) = F_r(g)\circ F_l(f)$.
    \end{enumerate}
  \end{itemize}

  \begin{enumerate}
  \item Construct a bijection between separately functorial and jointly functorial bifunctors $\cat C, \cat D \to \cat E$.
  \item Construct a bijection between jointly functorial bifunctors
    $\cat C, \cat D \to \cat E$ and functors $\cat C \times \cat D \to
    \cat E$.
  \item Construct a bijection between separately functorial bifunctors
    $\cat C, \cat D \to \cat E$ and functors $\cat C \to \cat E^{\cat
    D}$, where $\cat E^{\cat D}$ is the category of functors from
    $\cat D$ to $\cat E$ with natural transformations as morphisms.
  \end{enumerate}
\end{problem}

\begin{problem}{Product Functor}
  Let $\mathcal C$ be a category such that for every pair of objects
  $A,B \in \mathcal C$, we have a specified product $(A \times B ,
  \pi_1 : \mathcal C(A\times B, A), \pi_2 : \mathcal C(A\times B,B))$.
  \begin{enumerate}
  \item Show that taking binary products defines a functor $\times :
    (\cat C \times \cat C) \to \cat C$. That is, show that if we define
    $\times$ on objects such that $a \times b$ is a product of $a$ and
    $b$ (with projections $\pi_1 : a\times b \to a$ and $\pi_2 : a
    \times b \to b$), then you can extend the definition to a
    functorial action on morphisms.
  \item Let $\Pi_1 : \mathcal C^2 \to \mathcal C$ be the functor that
    projects out the first component of $\mathcal C^2$.  Prove that
    $\pi_1$ defines a natural transformation from $\times$ to $\Pi_1$.
    Symmetrically, $\pi_2$ is also natural.
  \end{enumerate}
\end{problem}

\begin{problem}{Theorems for Free, Naturally}
  The naturality property of a natural transformation is such a strong
  condition that sometimes we can characterize all natural
  transformations between two fixed functors, and in many examples
  there are only finitely many.
  
  This has direct applications to programming. The reason is that in a
  pure polymorphic functional language, given type constructors $F$
  and $G$ that are functorial, all functions $F(X) \to G(X)$ that are
  polymorphic in $X$ denote natural transformations! Phil Wadler,
  building on John Reynold's theory of parametricity called these
  ``theorems for free'': just from the type of a polymorphic function,
  the naturality property gives you properties that hold for every
  function of that type (\cite{wadler,reynolds}).
  \begin{enumerate}
  \item Define a natural transformation from $\id_{\Set}$ to
    $\id_{\Set}$ and prove that it is the only such natural
    transformation.

  \item Let $\times : \Set \times \Set \to \Set$ be the functor you
    deifned in the previous problem and let $\times' : \Set \times
    \Set \to \Set$ be the functor with the arguments swapped $A
    \times' B = B \times A$.

    Define a natural transformation from $\times$ to $\times'$ and
    show that it is the only such natural transformation.

  \item Recall the category of pointed sets $\Set_*$ is defined as follows:
    \begin{itemize}
    \item Objects are pairs of a set $X$ and a ``basepoint'' $x_0 \in X$.
    \item A morphism from $(X,x_0)$ to $(Y, y_0)$ is a
      \emph{base-point-preserving} function, i.e., a function $f : X
      \to Y$ such that $f(x_0) = y_0$. Identity and composition are
      simply identity and composition of functions.
    \end{itemize}

    Define two different natural transformations from
    $\id_{\Set_*}$ to $\id_{\Set_*}$ and prove that these
    are the only two such natural transformations.
  \end{enumerate}
\end{problem}
\newpage
\bibliography{cats}

\end{document}
